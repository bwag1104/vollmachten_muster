\documentclass[pdftex,12pt,a4paper]{article}

\usepackage[left=2.5cm,right=2.5cm,top=3cm,bottom=2cm,headheight=30pt]{geometry} 

\usepackage[utf8]{inputenc}
\usepackage[T1]{fontenc}
\usepackage{marvosym} % For smileys
\usepackage[main=german,english,french,latin]{babel}
\usepackage[style=german]{csquotes}    % Anführungszeichen \enquote{text}
%\usepackage{tex4ht}

\usepackage{parskip}

\usepackage{zref-totpages} % Gesamtzahl Seiten

\usepackage{amsmath}
\usepackage{amsfonts}
\usepackage{amssymb}
\usepackage{fancyhdr}
\usepackage{paralist}
\usepackage{url}
\usepackage{listings} \lstset{numbers=left, numberstyle=\tiny, numbersep=5pt} \lstset{language=HTML} 
 
 

\usepackage{enumitem}    % numbered lists
  
\usepackage[small]{titlesec}  % small section titles
\usepackage[flushmargin]{footmisc}  % footnotes aligned

\usepackage{helvet}  % Helvetica
\renewcommand{\familydefault}{\sfdefault}


\setlength{\parindent}{0pt}  % no indentation of paragraphs






\newcommand\emptyline{%
  \vspace{\baselineskip}
}   % Leerzeile

\newcommand\el{\emptyline}   % Leerzeile (Kurzform)


\newcommand\inserthrule{{\noindent\normalfont\rule{\textwidth}{1pt}}}
 % durchgehender waagerechter Strich


\newcommand{\leadingzero}[1]{\ifnum #1<10 0\the#1\else\the#1\fi}
   \renewcommand{\today}{\leadingzero{\day}.\leadingzero{\month}.\the\year}
 % Ändert das Datumsformat des Befehls \today von TT.MM.JJ in TT.MM.JJJJ



\newcommand{\gesamtzahlseiten}{}
\renewcommand{\gesamtzahlseiten}{\ (von insgesamt~\ztotpages )}

      

\newcounter{empfehlungsnummer}
\setcounter{empfehlungsnummer}{1}
\newcounter{empfehlungsnummer_lokal}
\setcounter{empfehlungsnummer_lokal}{1}

 %  \setcounter{empfehlungsnummer_lokal}{\value{empfehlungsnummer}}
   %  \usecounter{empfehlungsnummer_lokal}
   %  \settowidth\labelwidth{\makelabel{E--999}}
   %  \setlength\leftmargin{\labelwidth+\labelsep}
   
   
 %  \setlength{\marginempfehlungen}{3cm}
   
  
\newenvironment{empfehlung_einzel}
  {%

    \inserthrule  
  
   \noindent{\bf Empfehlung} 
  % \stepcounter{Empfehlungen}   
  \begin{enumerate}[resume*=empfehlungen]{%
    } 
}{%
     \end{enumerate}   
  

   \inserthrule
   
 }    
   
  
\newenvironment{empfehlungsliste}
  {%

  \inserthrule  
  
   \noindent{\bf Empfehlungen} 
  % \stepcounter{Empfehlungen}   
  \begin{enumerate}[resume*=empfehlungen]{%
    } 
}{%
     \end{enumerate}   
  

   \inserthrule
   
 }    
   
   
   
   
\newcommand{\myclearpage}{\clearpage} % either clearpage or empty
   
   

\newcommand\makedraft{%
  \usepackage{draftwatermark}
  \SetWatermarkText{ENTWURF \today}
  \SetWatermarkLightness{0.9}  % 1.0 white 0.0 black
  \SetWatermarkScale{1.5}    % Default 1.2
}


\newcommand\signaturefield{%

~

\begin{tabular}{ll}

\textit{Ort und Datum} & \rule[-0.2cm]{10cm}{.3mm} \\

~  & ~ \\

~ & ~ \\

\textit{Unterschrift} &  \rule[-0.2cm]{10cm}{.3mm} \\

\end{tabular}

}




\newcommand{\mysign}{wu}
\newcommand{\myfirstname}{Wolfgang}
\newcommand{\mylastname}{Uebel}
\newcommand{\mybirthday}{12.\ April 1942}
\newcommand{\myplaceofbirth}{Mannheim}
\newcommand{\mystreet}{Schriesheimer Fußweg 20}
\newcommand{\mycity}{68526 Ladenburg}
\newcommand{\myemail}{wolf.uebel@web.de}
\newcommand{\mymobile}{+49 6203 128 37}


\newcommand{\dateissued}{25.\ Januar 2017}

\newcommand{\mytrustedone}{Frau Dr.\ Annette Wagemann}
\newcommand{\mytrustedonebirthday}{07.\ November 1963}
\newcommand{\mytrustedonestreet}{Schriesheimer Fußweg 20}
\newcommand{\mytrustedonecity}{68526 Ladenburg}
\newcommand{\mytrustedoneemail}{a.wagem@web.de}
\newcommand{\mytrustedonemobile}{+49 152 327 96939}

\newcommand{\mytrustedtwo}{Herr Tim Wagemann}
\newcommand{\mytrustedtwobirthday}{08.\ Juli 2000}
\newcommand{\mytrustedtwostreet}{Schriesheimer Fußweg 20}
\newcommand{\mytrustedtwocity}{68526 Ladenburg}
\newcommand{\mytrustedtwoemail}{tim.wag@outlook.de}
\newcommand{\mytrustedtwomobile}{+49 176 436 16144}

\newcommand{\mytrustedthree}{Frau Dr. med. Anna Kühlert}
\newcommand{\mytrustedthreestreet}{Cornel-Serr-Platz 4}
\newcommand{\mytrustedthreecity}{68526 Ladenburg}
\newcommand{\mytrustedthreeemail}{info@praxis-fuerstenberg.de}
\newcommand{\mytrustedthreemobile}{+49 6203 181018}

% to be included after the my_data_xx file

\newcommand{\mydataofbirth}{geboren am \mybirthday\ in \myplaceofbirth}
\newcommand{\myaddress}{\mystreet, \mycity}
\newcommand{\myfullname}{\myfirstname\ \mylastname}
\newcommand{\mydateofbirth}{\mybirthday}

\newcommand{\mytrustedoneaddress}{\mytrustedonestreet, \mytrustedonecity}
\newcommand{\mytrustedonedataofbirth}{geboren am \mytrustedonebirthday\ in \mytrustedoneplaceofbirth}

\newcommand{\dateprinted}{22.\ Februar 2017}


% \makedraft

\author{\myfullname}
\title{Vorsorgevollmacht für \\
       \mytrustedone\\
       }
\date{\dateissued}







% ============= Kopf- und Fußzeile =============
\pagestyle{fancy}
%
\lhead{\small \myfullname}
\chead{}
\rhead{\small Vorsorgevollmacht für \mytrustedone} 
%
\lfoot{\small Fassung vom \dateissued, gedruckt am \dateprinted}
\cfoot{}
\rfoot{\small Seite \thepage\gesamtzahlseiten}
%

\renewcommand{\footrulewidth}{0.4pt} 

\usepackage{pdfpages}                 % Für \includepdf{.pdf}
\setlist[enumerate]{label=\arabic*.\ ,leftmargin=1cm}

\newlist{legal}{enumerate}{10}
\setlist[legal]{label=(\arabic*)\ ,leftmargin=1cm}



% Dokumentinformationen
\usepackage{hyperref}                  % \url{URL} \href{URL}{text}



\hypersetup{
   unicode=true,
   bookmarksopenlevel={section},
   pdftoolbar=true,
   pdftitle={vorsorgevollmacht_\mysign},
   pdfsubject={Vorsorgevollmacht}
   pdfauthor={\myfullname},
   pdfkeywords={Vorsorgevollmacht},
   pdfcreator={pdflatex},
   colorlinks=true,
   breaklinks=true,
   citecolor={black},
   linkcolor={blue},  
   menucolor={black},  
   urlcolor={blue},
   draft=false
   }


\begin{document}







\hyphenation{
  ärzt-lich
  ärzt-li-che
  Be-hand-lung
  Be-rufs-un-fä-hig-keit
  Be-rufs-un-fä-hig-keits-ren-te
  Dauer-be-wusst-los-ig-keit
  De-velop-ment
  durch-ge-führt
  Fä-hig-keit
  Fin-ger-spi-tzen-ge-fühl
  fühl-bar
  fühl-ba-ren
  Ge-sund-heit
  ge-sund-heit-lich
  ge-sund-heit-li-chen
  Ge-sund-heits-zu-stand
  Ge-sund-heits-zu-stands
  Hand-lung
  Heil-be-hand-lung
  In-halts-ver-ständ-nis
  In-halts-ver-zeich-nis
  Kran-ken-ver-si-che-rung
  Kran-ken-ver-si-che-rungs
  Kran-ken-ver-si-che-rungen
  Le-bens-ver-si-che-rung
  Le-bens-ver-si-che-rungs
  Mo-ti-va-tion
  Mo-ti-va-tions-fä-hig-keit
  Pro-zess-ko-sten-finan-zierung
  Re-ge-lung
  Re-ge-lun-gen
  Sach-the-men
  Schul-ter-klo-pfen
  Si-tu-ations-be-schrei-bung
  Si-tu-ations-be-schrei-bungen
  Symp-tom
  Symp-tom-be-hand-lung
  Übungs-typen
  Ver-sand
  Ver-siche-rung
  Vor-stands-as-sis-tent
  wert-schöp-fend
  Wert-schöp-fung
  Wie-der-be-le-bung
  Wie-der-be-le-bungs-maß-nahme
  Wie-der-be-le-bungs-maß-nahmen
  Zwangs-maß-nah-me
  Zwangs-maß-nah-men
}




% \bibliographystyle{alpha}



\maketitle
\tableofcontents
\setcounter{page}{1}

\newpage




Ich,

\myfullname,

\mydataofbirth,

derzeit wohnhaft in \myaddress

-- nachstehend \enquote{Vollmachtgeber} genannt -- 

~


erteile hiermit Vollmacht an

~
 
\mytrustedone, geboren am \mytrustedonebirthday,

derzeit wohnhaft in \mytrustedoneaddress,

-- nachstehend als \enquote{bevollmächtigte Person} oder 
\enquote{Vertrauensperson} oder \enquote{sie} bezeichnet.

~

Diese Vertrauensperson wird hiermit bevollmächtigt, mich in allen Angelegenheiten zu vertreten, die ich
im Folgenden angegeben habe. Durch diese Vollmachterteilung soll eine vom Gericht
angeordnete Betreuung vermieden werden. Die Vollmacht bleibt daher in Kraft, wenn ich nach ihrer
Errichtung geschäftsunfähig geworden sein sollte.

Die Vollmacht ist nur wirksam, solange die bevollmächtigte Person die Vollmachtsurkunde besitzt und
bei Vornahme eines Rechtsgeschäfts die Urkunde im Original vorlegen kann.

Aus Gründen der Praktikabilität kann es mehrere Originalurkunden geben.

Die bevollmächtigte Person kann diese Vollmacht alleine ausüben. Dem steht nicht entgegen, dass ich
weitere Vollmachten erteilt habe oder dies tun werde. 

Die Kontaktdaten der hier genannten Personen werden auf einem separaten Beiblatt erfasst. Damit sind sie leichter aktualisierbar.


\section{Gesundheitssorge und Pflegebedürftigkeit}

Sie darf in allen Angelegenheiten der Gesundheitssorge entscheiden, ebenso über
alle Einzelheiten einer ambulanten oder (teil-)stationären Pflege. Sie ist befugt,
meinen
in einer Patientenverfügung festgelegten Willen durchzusetzen.

 
 
Sie darf insbesondere in eine Untersuchung des Gesundheitszustands, eine
Heilbehandlung
oder einen ärztlichen Eingriff einwilligen, diese ablehnen oder die
Einwilligung in diese Maßnahmen widerrufen, auch wenn mit der Vornahme, dem
Unterlassen oder dem Abbruch dieser Maßnahmen die Gefahr besteht, dass ich
sterbe
oder einen schweren und länger dauernden gesundheitlichen Schaden erleide
(§ 1904 Absatz 1 und 2 BGB).

 
 
Sie darf Krankenunterlagen einsehen und deren Herausgabe an Dritte bewilligen.
Ich entbinde alle mich behandelnden Ärzte und nichtärztliches Personal gegenüber
meiner bevollmächtigten Vertrauensperson von der Schweigepflicht. Diese darf ihrerseits
alle mich behandelnden Ärzte und nichtärztliches Personal von der Schweigepflicht
gegenüber Dritten entbinden.

 
 
Solange es zu meinem Wohl erforderlich ist, darf sie

\begin{itemize}

\item über meine freiheitsentziehende Unterbringung (§ 1906 Absatz 1 BGB), 

\item über freiheitsentziehende Maßnahmen (z.B.\ Bettgitter, Medikamente u.\ ä.) in einem Heim oder in einer sonstigen Einrichtung (§ 1906 Absatz 4 BGB),

\item über ärztliche Zwangsmaßnahmen (§ 1906a Absatz 1 BGB),

\item über meine Verbringung zu einem stationären Aufenthalt in einem Krankenhaus, wenn eine ärztliche Zwangsmaßnahme in Betracht kommt (§ 1906a Absatz 4 BGB), 


\end{itemize}

entscheiden.

\section{Aufenthalt und Wohnungsangelegenheiten} 

Sie darf meinen Aufenthalt bestimmen. 

Sie darf Rechte und Pflichten aus dem Vertrag über mein Haus/Wohnung
(einschließlich Veräußerung, Kündigung) wahrnehmen sowie meinen Haushalt auflösen. 
Sie darf neue Verträge abschließen und kündigen.

 
 
Sie darf Verträge nach dem Wohn- und Betreuungsvertragsgesetz
(Vertrag über die Überlassung von Wohnraum mit Pflege- und Betreuungsleistungen; ehemals: Heimvertrag) abschließen und kündigen. 


\section{Behörden}

Sie darf mich bei Behörden, Versicherungen, Renten- und Sozialleistungsträgern vertreten. Dies umfasst auch die 
datenschutzrechtliche Einwilligung.

 

\section{Vermögenssorge}

Sie darf mein Vermögen verwalten und hierbei alle Rechtshandlungen und
Rechtsgeschäfte im In- und Ausland vornehmen, Erklärungen aller Art abgeben
und entgegennehmen sowie Anträge stellen, abändern, zurücknehmen, 

namentlich darf sie

\begin{itemize}

\item über Vermögensgegenstände jeder Art verfügen,
 
 \item Auskunft von Banken, Versicherungen, Investmentgesellschaften
 sowie allen Institutionen, die für mich zur Vermögensmehrung 
 bzw.\ -verwaltung tätig sind, erbitten und erhalten,

\item Zahlungen und Wertgegenstände annehmen,

\item Verbindlichkeiten eingehen,

\item Willenserklärungen bezüglich meiner Konten, Depots und Safes abgeben. Sie darf
mich im Geschäftsverkehr mit Kreditinstituten vertreten,

\item Schenkungen in dem Rahmen vornehmen, der einem Betreuer rechtlich gestattet ist. 

\end{itemize}



\section{Post und Fernmeldeverkehr}

Sie darf im Rahmen der Ausübung dieser Vollmacht die für mich bestimmte Post
entgegennehmen, öffnen und lesen. Dies gilt auch für den elektronischen Postverkehr.

Zudem darf sie über den Fernmeldeverkehr einschließlich aller elektronischen
Kommunikationsformen entscheiden. Sie darf alle hiermit zusammenhängenden
Willenserklärungen (z. B. Vertragsabschlüsse, Kündigungen) abgeben. 

\section{Steuererklärungen}

Auch darf sie meine Steuererklärungen abgeben sowie die Bescheide 
entgegennehmen.

\section{Vertretung vor Gericht} 

Sie darf mich gegenüber Gerichten vertreten sowie Prozesshandlungen aller Art
vornehmen. 

\section{Weitere Regelungen}

Sie darf Untervollmacht erteilen. 

Falls trotz dieser Vollmacht eine gesetzliche Vertretung (\enquote{rechtliche Betreuung}) erforderlich
sein sollte, bitte ich, die oben bezeichnete Vertrauensperson als Betreuer zu bestellen. 

Die Vollmacht gilt über den Tod hinaus. 





~

Vollmachtnehmer: Ich bin einverstanden.

\mytrustedone

\signaturefield

~

Vollmachtgeber: Gemäß diesen Bestimmungen erteile ich Vollmacht.

\myfullname

\signaturefield




\end{document}
