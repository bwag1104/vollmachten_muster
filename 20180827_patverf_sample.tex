\documentclass[pdftex,12pt,a4paper]{article}

\usepackage[left=2.5cm,right=2.5cm,top=3cm,bottom=2cm,headheight=30pt]{geometry} 

\usepackage[utf8]{inputenc}
\usepackage[T1]{fontenc}
\usepackage{marvosym} % For smileys
\usepackage[main=german,english,french,latin]{babel}
\usepackage[style=german]{csquotes}    % Anführungszeichen \enquote{text}
%\usepackage{tex4ht}

\usepackage{parskip}

\usepackage{zref-totpages} % Gesamtzahl Seiten

\usepackage{amsmath}
\usepackage{amsfonts}
\usepackage{amssymb}
\usepackage{fancyhdr}
\usepackage{paralist}
\usepackage{url}
\usepackage{listings} \lstset{numbers=left, numberstyle=\tiny, numbersep=5pt} \lstset{language=HTML} 
 
 

\usepackage{enumitem}    % numbered lists
  
\usepackage[small]{titlesec}  % small section titles
\usepackage[flushmargin]{footmisc}  % footnotes aligned

\usepackage{helvet}  % Helvetica
\renewcommand{\familydefault}{\sfdefault}


\setlength{\parindent}{0pt}  % no indentation of paragraphs






\newcommand\emptyline{%
  \vspace{\baselineskip}
}   % Leerzeile

\newcommand\el{\emptyline}   % Leerzeile (Kurzform)


\newcommand\inserthrule{{\noindent\normalfont\rule{\textwidth}{1pt}}}
 % durchgehender waagerechter Strich


\newcommand{\leadingzero}[1]{\ifnum #1<10 0\the#1\else\the#1\fi}
   \renewcommand{\today}{\leadingzero{\day}.\leadingzero{\month}.\the\year}
 % Ändert das Datumsformat des Befehls \today von TT.MM.JJ in TT.MM.JJJJ



\newcommand{\gesamtzahlseiten}{}
\renewcommand{\gesamtzahlseiten}{\ (von insgesamt~\ztotpages )}

      

\newcounter{empfehlungsnummer}
\setcounter{empfehlungsnummer}{1}
\newcounter{empfehlungsnummer_lokal}
\setcounter{empfehlungsnummer_lokal}{1}

 %  \setcounter{empfehlungsnummer_lokal}{\value{empfehlungsnummer}}
   %  \usecounter{empfehlungsnummer_lokal}
   %  \settowidth\labelwidth{\makelabel{E--999}}
   %  \setlength\leftmargin{\labelwidth+\labelsep}
   
   
 %  \setlength{\marginempfehlungen}{3cm}
   
  
\newenvironment{empfehlung_einzel}
  {%

    \inserthrule  
  
   \noindent{\bf Empfehlung} 
  % \stepcounter{Empfehlungen}   
  \begin{enumerate}[resume*=empfehlungen]{%
    } 
}{%
     \end{enumerate}   
  

   \inserthrule
   
 }    
   
  
\newenvironment{empfehlungsliste}
  {%

  \inserthrule  
  
   \noindent{\bf Empfehlungen} 
  % \stepcounter{Empfehlungen}   
  \begin{enumerate}[resume*=empfehlungen]{%
    } 
}{%
     \end{enumerate}   
  

   \inserthrule
   
 }    
   
   
   
   
\newcommand{\myclearpage}{\clearpage} % either clearpage or empty
   
   

\newcommand\makedraft{%
  \usepackage{draftwatermark}
  \SetWatermarkText{ENTWURF \today}
  \SetWatermarkLightness{0.9}  % 1.0 white 0.0 black
  \SetWatermarkScale{1.5}    % Default 1.2
}


\newcommand\signaturefield{%

~

\begin{tabular}{ll}

\textit{Ort und Datum} & \rule[-0.2cm]{10cm}{.3mm} \\

~  & ~ \\

~ & ~ \\

\textit{Unterschrift} &  \rule[-0.2cm]{10cm}{.3mm} \\

\end{tabular}

}




\newcommand{\mysign}{bb}
\newcommand{\myfirstname}{Benjamin}
\newcommand{\mylastname}{Beispiel}
\newcommand{\mybirthday}{12.\ Mai 1965}
\newcommand{\myplaceofbirth}{Köln}
\newcommand{\mystreet}{Grasserstr.\ 10}
\newcommand{\mycity}{80333 München}
\newcommand{\myemail}{b.beispiel@web.de}
\newcommand{\mymobile}{+49 152 333 0815}


\newcommand{\dateissued}{11.\ August 2018}

\newcommand{\mytrustedone}{Frau Dr.\ Caroline Chassidy}
\newcommand{\mytrustedonebirthday}{11.\ November 1968}
\newcommand{\mytrustedoneplaceofbirth}{Mannheim}
\newcommand{\mytrustedonestreet}{An der krummemn Klenke 20}
\newcommand{\mytrustedonecity}{10234 Berlin}
\newcommand{\mytrustedoneemail}{c.c@web.de}
\newcommand{\mytrustedonemobile}{+49 152 328 4712}

\newcommand{\mytrustedtwo}{Herr Tim Tartek}
\newcommand{\mytrustedtwobirthday}{23.\ Juli 2002}
\newcommand{\mytrustedtwoplaceofbirth}{Hamburg}
\newcommand{\mytrustedtwostreet}{Segelstr.\ 28}
\newcommand{\mytrustedtwocity}{20540 Kiel}
\newcommand{\mytrustedtwoemail}{tim.tartek@outlook.de}
\newcommand{\mytrustedtwomobile}{+49 176 436 47892}

\newcommand{\mytrustedthree}{Frau Dr.\ Daisy Belle}
\newcommand{\mytrustedthreestreet}{Grünspanplatz 12}
\newcommand{\mytrustedthreecity}{70173 Stuttgart}
\newcommand{\mytrustedthreeemail}{info@daisy-belle.de}
\newcommand{\mytrustedthreemobile}{+49 711 4521 4713}

% to be included after the my_data_xx file

\newcommand{\mydataofbirth}{geboren am \mybirthday\ in \myplaceofbirth}
\newcommand{\myaddress}{\mystreet, \mycity}
\newcommand{\myfullname}{\myfirstname\ \mylastname}
\newcommand{\mydateofbirth}{\mybirthday}

\newcommand{\mytrustedoneaddress}{\mytrustedonestreet, \mytrustedonecity}
\newcommand{\mytrustedonedataofbirth}{geboren am \mytrustedonebirthday\ in \mytrustedoneplaceofbirth}


% \makedraft

\author{\myfullname}
\title{Patientenverfügung\\
       }
\date{\dateissued}





% ============= Kopf- und Fußzeile =============
\pagestyle{fancy}
%
\lhead{\small \myfullname}
\chead{}
\rhead{\small Patientenverfügung} 
%
\lfoot{\small Fassung \dateissued, gedruckt \today}
\cfoot{}
\rfoot{\small Seite \thepage\gesamtzahlseiten}
%

\renewcommand{\footrulewidth}{0.4pt} 

\usepackage{pdfpages}                 % Für \includepdf{.pdf}
\setlist[enumerate]{label=E--\arabic*,leftmargin=2cm}


% Dokumentinformationen
\usepackage{hyperref}                  % \url{URL} \href{URL}{text}



\hypersetup{
   unicode=true,
   bookmarksopenlevel={section},
   pdftoolbar=true,
   pdftitle={patverf_ww},
   pdfsubject={Patientenverfügung},
   pdfauthor={\myfullname},
   pdfkeywords={Patientenverfügung, Morbus Parkinson},
   pdfcreator={pdflatex},
   colorlinks=true,
   breaklinks=true,
   citecolor={black},
   linkcolor={blue},  
   menucolor={black},  
   urlcolor={blue},
   draft=false
   }


\begin{document}







\hyphenation{
  ärzt-lich
  ärzt-li-che
  Be-hand-lung
  Be-rufs-un-fä-hig-keit
  Be-rufs-un-fä-hig-keits-ren-te
  Dauer-be-wusst-los-ig-keit
  De-velop-ment
  durch-ge-führt
  Fä-hig-keit
  Fin-ger-spi-tzen-ge-fühl
  fühl-bar
  fühl-ba-ren
  Ge-sund-heit
  ge-sund-heit-lich
  ge-sund-heit-li-chen
  Ge-sund-heits-zu-stand
  Ge-sund-heits-zu-stands
  Hand-lung
  Heil-be-hand-lung
  In-halts-ver-ständ-nis
  In-halts-ver-zeich-nis
  Kran-ken-ver-si-che-rung
  Kran-ken-ver-si-che-rungs
  Kran-ken-ver-si-che-rungen
  Le-bens-ver-si-che-rung
  Le-bens-ver-si-che-rungs
  Mo-ti-va-tion
  Mo-ti-va-tions-fä-hig-keit
  Pro-zess-ko-sten-finan-zierung
  Re-ge-lung
  Re-ge-lun-gen
  Sach-the-men
  Schul-ter-klo-pfen
  Si-tu-ations-be-schrei-bung
  Si-tu-ations-be-schrei-bungen
  Symp-tom
  Symp-tom-be-hand-lung
  Übungs-typen
  Ver-sand
  Ver-siche-rung
  Vor-stands-as-sis-tent
  wert-schöp-fend
  Wert-schöp-fung
  Wie-der-be-le-bung
  Wie-der-be-le-bungs-maß-nahme
  Wie-der-be-le-bungs-maß-nahmen
  Zwangs-maß-nah-me
  Zwangs-maß-nah-men
}




% \bibliographystyle{alpha}



\maketitle
\tableofcontents
\setcounter{page}{1}

\newpage


\section{Kurzübersicht}


\subsection{Personen}

\begin{tabular}{ll}

\textbf{Personen} & ~ \\


~  & ~ \\

Verfügender Patient & \myfullname\ \\
~                   &  geb.\ am \mybirthday\ \\


~  & ~ \\



1. Vertrauensperson & \mytrustedone\ \\
und Bevollmächtigte  & geb.\ am \mytrustedonebirthday\ \\


~  & ~ \\


2. Vertrauensperson & \mytrustedtwo\ \\
und Bevollmächtigter  & geb.\ am \mytrustedtwobirthday\ \\


~  & ~ \\


3. Vertrauensperson & \mytrustedthree\ \\
behandelnde Neurologin  & ~ \\


~  & ~ \\

Kontaktdaten für alle & Siehe Beiblatt \\

~ & ~ \\

\end{tabular}

\subsection{Situationen}

\begin{itemize}

\item Wahrscheinlich unabwendbarer Sterbeprozess

\item Endstadium einer unheilbaren, tödlichen Krankheit

\item Koma länger als 1 Woche

\item Praktisch vollkommene Bewegungsunfähigkeit

\item Schwere Impuls-Kontrollstörungen (Beurteilung durch einen Neurologen)

\item Erloschene Fähigkeit zu Entscheidungen und Kontakt

\item  Unfähigkeit zur Aufnahme von Nahrung und Flüssigkeit (z.B.\ bei Demenz)


\end{itemize}


\subsection{Medizinische Maßnahmen}

\begin{tabular}{ll}



Lebenserhaltende Maßnahmen & nein \\

~  & ~ \\


Schmerz- und  & nur im Rahmen einer \\
Symptombehandlung & palliativmedizinischen Versorgung \\

~  & ~ \\


Künstliche Ernährung und & nur bei palliativmedizinischer Indikation \\
Flüssigkeitszufuhr &  zur Beschwerdelinderung \\

~  & ~ \\


Wiederbelebung und & keine \\
künstliche Beatmung & ~ \\

~  & ~ \\


Dialyse & keine \\

~  & ~ \\


Gabe von Blut oder & nur bei palliativmedizinischer Indikation \\ 
Blutbestandteilen & zur Beschwerdelinderung \\

~  & ~ \\


Antibiotika & nur bei palliativmedizinischer Indikation \\
~ &  zur Beschwerdelinderung \\

~  & ~ \\


Sonstige Medikation & Parkinson-Medikamente sollen \\
~ & weiterhin gegeben  \\

~ & und im Bedarfsfall auf Geheiß \\
~ & eines Neurologen angepasst werden \\

~  & ~ \\


Operationen & keine, auch keine Tiefe-Hirnstimulation (THS) \\ 

~  & ~ \\




\end{tabular}


\subsection{Regelungen für den Todesfall}

\begin{itemize}

\item Verbrennung des Leichnams

\item Einwilligung zur Organspende -- allerdings vorliegende Erkrankung
Morbus Parkinson: Organe vermutlich mit Medikamenten belastet

\end{itemize}


\newpage


\section{Grundüberlegungen}

Ich habe keine Angst vor dem Tod. Beim Sterben dagegen ist meine 
Befürchtung, dass es lang und qualvoll sein könnte. Das will ich natürlich nicht.

Und ebensowenig möchte ich ein Leben führen, das dauerhaft vom
Funktionieren einer Maschine abhängt.  

Ich bin in der glücklichen Position, alles, was ich im Leben
erreichen wollte, auch erreicht zu haben. 

Mit diesen Gedankengängen erschließt sich die Logik der Entscheidungen 
sehr schnell.

Ich bitte um Abschaltung von Maschinen, an denen mein Leben hängt, 
es sei denn eine solche Maschine hätte bereits vorher 
mein Leben bestimmt. Und in gleicher Weise will ich auch keine
andere Maßnahme: keine Wiederbelebung, 
keine künstliche Beatmung, keine Dialyse etc.
 
Von dieser Grundregel gibt es nur wenige Ausnahmen. Das sind:

\begin{itemize}

\item Schmerzen: Leider bin ich ziemlich empfindlich. Deshalb 
bitte ich,
Schmerzen zu dämpfen bzw.\ keine Maßnahmen zu ergreifen, die große 
Schmerzen auslösen.

\item Parkinson-bezogene Maßnahmen: Diese können normal
fortgesetzt werden. Deshalb bleiben die Medikamente
so wie sie sind.

\end{itemize}

Sollte deshalb jemals der Fall eintreten, dass eine Entscheidung 
auf Basis dieser Patientenverfügung zu treffen ist, dann
sollte vor allem eins möglichst schnell geklärt werden:
Kann ich es ohne Ankopplung an eine Maschine weiter schaffen?

Wenn die Antwort darauf \enquote{nein} lautet, dann lassen 
Sie die Maschine bitte abschalten -- ohne schlechtes Gewissen.





\newpage


\section{Eingangsformel}


Ich, 

\myfullname,

geboren am \mybirthday\ in \myplaceofbirth,  

derzeit wohnhaft in \myaddress, 
 
\textit{(Name, Vorname, geboren am, wohnhaft in)}

bestimme hiermit für den Fall, dass ich meinen Willen nicht mehr bilden oder verständlich äußern kann, in den nachfolgend aufgezählten \enquote{Situationen der Anwendung} die Durchführung 
der Behandlung wie im Abschnitt \enquote{Medizinische Behandlung
und Pflege} beschrieben.\footnote{Sprachlicher Hinweis: Die grammatikalisch maskuline Form
umfasst auch das reale weibliche Geschlecht. Deshalb spreche ich z.B. nicht
von \enquote{dem Arzt/der Ärztin}. Sondern ich verwende nur die
maskuline Form und meine damit weibliche wie männliche Ärzte. Dem steht nicht entgegen, dass bei klar bestimmten Personen das Geschlecht auch zum 
Ausdruck gebracht werden kann.}


\section{Situationen der Anwendung}


Diese Regelungen sollen in den nachfolgend exemplarisch beschriebenen Situationen zur Anwendung kommen. Mit \enquote{exemplarisch} ist gemeint, dass diese Aufzählung nicht vollständig ist, sondern nur beispielhaft. Bei Vorliegen ähnlicher Gegebenheiten soll also ebenso verfahren werden.

Die Situationen sind diese:

\begin{itemize}


\item	Ich befinde mich aller Wahrscheinlichkeit nach unabwendbar im unmittelbaren Sterbeprozess.


\item	Ich befinde mich im Endstadium einer unheilbaren, tödlich verlaufenden Krankheit, egal ob der Todeszeitpunkt absehbar ist oder nicht.


\item	Ich liege länger als 1 Woche im Koma oder bin in diesem Zeitraum bewusstlos.

\item	Ich bin -- zum Beispiel bedingt durch Parkinson\footnote{Weitere Bezeichnungen sind: \textit{Morbus Parkinson}, Parkinson'sche Erkrankung und ähnliche, Abkürzung: \textit{MP}}, eine Krankheit,
an der ich leide -- praktisch vollkommen bewegungsunfähig.


\item	Ich habe dermaßen häufige und schwere Impuls-Kontrollstörungen, dass
das Treffen weitreichender Entscheidungen durch mich besser unterbleiben sollte. Die
Beurteilung sollte durch einen Neurologen erfolgen.

\item	Meine Fähigkeit, Einsichten zu gewinnen, Entscheidungen zu treffen und mit anderen Menschen in Kontakt zu treten, ist aufgrund einer Hirnschädigung aller Wahrscheinlichkeit nach unwiederbringlich erloschen. Dies gilt für direkte Gehirnschädigung z.B.\ durch Unfall, Schlaganfall oder Entzündung ebenso wie für indirekte Gehirnschädigung z.B.\ nach Wiederbelebung, Schock oder Lungenversagen. Es ist mir bewusst, dass in solchen Situationen die Fähigkeit zu Empfindungen erhalten sein kann und dass ein Aufwachen aus diesem Zustand nicht ganz sicher auszuschließen, aber unwahrscheinlich ist.

 Ob die Entscheidungsfähigkeit in diesem Sinne endgültig verloren ist, soll anhand der Einschätzung zweier Ärzte beurteilt werden.


\item	Ich bin infolge eines weit fortgeschrittenen Hirnabbauprozesses
(z. B. bei Demenzerkrankung) auch mit ausdauernder Hilfestellung
nicht mehr in der Lage, Nahrung und Flüssigkeit auf natürliche Weise zu mir zu nehmen.

 
 
\end{itemize}
 
 
 
\section{Medizinische Behandlung und Pflege}



\subsection{Lebenserhaltende Maßnahmen}

In den oben beschriebenen Situationen wünsche ich,
dass alle lebenserhaltenden Maßnahmen \textbf{unterlassen werden}. 

Hunger und Durst sollen auf natürliche Weise gestillt werden, gegebenenfalls mit Hilfe bei der Nahrungs- und Flüssigkeitsaufnahme. Ich wünsche Pflege von Mund und Schleimhäuten sowie Körperpflege und das Lindern von Schmerzen, Atemnot, Übelkeit, Angst, Unruhe, Krämpfen, Schüttelfrost sowie 
auftretender Parkinson-Symptome.

\subsection{Schmerz- und Symptombehandlung}

In den oben beschriebenen Situationen wünsche ich eine 
Schmerz- und Symptombehandlung,
wenn alle sonstigen medizinischen Möglichkeiten zur Schmerz- und Symptomkontrolle versagen, auch Mittel mit bewusstseinsdämpfenden Wirkungen zur Beschwerdelinderung.

Sollte bei einer gewählten Medikation ein geringes Risiko 
bestehen, dass diese zu einer Verkürzung der Lebenszeit führt, so
nehme ich dieses in Kauf.



\subsection{Künstliche Ernährung und Flüssigkeitszufuhr}

In den oben beschriebenen Situationen wünsche ich,
dass eine künstliche Ernährung oder eine künstliche Flüssigkeitszufuhr
\textbf{nur bei palliativmedizinischer Indikation zur Beschwerdelinderung} erfolgen.


\subsection{Wiederbelebung und künstliche Beatmung}

In den oben beschriebenen Situationen sowie in allen Fällen eines
Kreislaufstillstands oder Atemversagens wünsche ich
\textbf{die Unterlassung von Versuchen der Wiederbelebung}.

Außerdem soll ein Notarzt \textbf{nicht verständigt} werden bzw.\ im Fall
einer Hinzuziehung unverzüglich über meine \textbf{Ablehnung von Wiederbelebungsmaßnahmen} informiert werden.


In den oben beschriebenen Situationen wünsche ich,
dass \textbf{keine künstliche Beatmung durchgeführt} bzw.\ 
 eine \textbf{schon eingeleitete Beatmung eingestellt} wird.
 Dies gilt unter der Voraussetzung, dass ich Medikamente zur Linderung der Luftnot erhalte. Die Möglichkeit einer Bewusstseinsdämpfung oder einer Verkürzung meiner Lebenszeit durch diese Medikamente nehme ich in Kauf.

\subsection{Dialyse, Blut, Blutbestandteile}

In den oben beschriebenen Situationen wünsche ich,
dass \textbf{keine Dialyse durchgeführt} bzw. eine \textbf{schon 
eingeleitete Dialyse eingestellt} wird.

In den oben beschriebenen Situationen wünsche ich
die Gabe von Blut oder Blutbestandteilen
\textbf{nur bei palliativmedizinischer Indikation zur Beschwerdelinderung}.
  
\subsection{Antibiotika und sonstige Medikation}

In den oben beschriebenen Situationen wünsche ich
Antibiotika \textbf{nur bei palliativmedizinischer Indikation zur Beschwerdelinderung}.

Die Gabe geeigneter Parkinson-Medikamente soll solange erfolgen, bis
ich sterbe. Eine Änderung der Medikation soll vorgenommen werden,
wenn ein Neurologe dies als sinnvoll erachtet.

Beispielsweise sollte ein Dopaminagonist abgesetzt oder reduziert werden, sofern er bei mir Impuls-Kontroll-Störungen auslöst.

Und auch weitere Medikamente sollen gegeben werden, sofern sie einen
fühlbaren Mehrwert stiften. Zum Beispiel sollten krampflösende 
Mittel gegeben werden, wenn Krämpfe auftreten (was in fortgeschrittenem Parkinson-Stadium nicht selten geschieht).


\subsection{Operationen}

Ich wünsche \textbf{keine} Operationen.

Dies gilt auch und insbesondere für die sogenannte \enquote{Tiefe
Hirnstimulation} (THS, auch oft nach dem englischen Begriff 
\textit{deep brain stimulation} als \textit{dbs} abgekürzt).



\subsection{Ort der Behandlung, Beistand}

Ich habe diesbezüglich keine besonderen Wünsche.


\section{Umgang mit der Verfügung}

\subsection{Besondere Vertrauenspersonen und Bevollmächtigte}

Für einige der nachfolgenden Abschnitte müssen Personen benannt werden, 
denen ich voll vertraue. Ich benenne diese hier:

\begin{itemize}


\item[1.] \mytrustedone, geb.\ am \mytrustedonebirthday,

\item[2.] \mytrustedtwo. geb.\ am \mytrustedtwobirthday.


\end{itemize}

Diese beiden erhalten zugleich -- jeweils einzeln -- alle erforderlichen Vollmachten, um mich in Gesundheitsbelangen zu vertreten.

Außerdem gehört zum Personenkreis der benannten Vertrauenspersonen 
die Neurologin, die mich über viele Jahre hinweg sehr gut behandelt hat:


\begin{itemize}

\item[3.] \mytrustedthree.


\end{itemize}

Ein Blatt mit den derzeitigen Kontaktdaten dieser Personen wird
separat beigefügt. Sie werden nachfolgend als \enquote{benannte
Vertrauenspersonen} bezeichnet.

Weitere Vertrauenspersonen können auf dem Kontaktdatenblatt
ergänzt werden. Dazu muss diese Verfügung nicht
geändert werden. Auch für sie gelten die Regelungen dieser
Verfügung. 

Anders dagegen ist es bei der Rückgängigmachung einer
Benennung als Vertrauensperson. Dies kann nur durch
Neufassung dieser Patientenverfügung durchgeführt werden.


\subsection{Entbindung von der ärztlichen Schweigepflicht}

Ich entbinde die mich behandelnden Ärzte von der Schweigepflicht 
gegenüber allen, die im Abschnitt \enquote{Besondere Vertrauenspersonen} benannt sind.


\subsection{Verbindlichkeit und Durchsetzung}

Der in meiner Patientenverfügung geäußerte Wille zu bestimmten ärztlichen und pflegerischen Maßnahmen soll von den behandelnden Ärzten und dem Behandlungsteam befolgt werden. Jede der von mir
als vertrauenswürdig benannten Personen (s.o.\ \enquote{Besondere Vertrauenspersonen}) soll sich dafür
einsetzen, dass mein Patientenwille durchgesetzt wird.

Sollte ein Arzt oder das Behandlungsteam nicht bereit sein, meinen in dieser Patientenverfügung geäußerten Willen zu befolgen, erwarte ich, dass für eine anderweitige medizinische oder pflegerische Behandlung gesorgt wird. Von den als vertrauenswürdig Benannten wünsche ich mir, dass sie die weitere Behandlung so organisieren, dass meinem Willen entsprochen wird.

\subsection{Auslegung}

Wenn ich meine Patientenverfügung nicht widerrufen habe, wünsche ich nicht, dass mir in der konkreten Anwendungssituation eine Änderung meines Willens unterstellt wird. Wenn aber die behandelnden Ärzte/das Behandlungsteam/die benannten Vertrauenspersonen aufgrund meiner Gesten, Blicke oder anderen Äußerungen die Auffassung vertreten, dass ich entgegen den Festlegungen in meiner Patientenverfügung doch behandelt oder nicht behandelt werden möchte, dann ist im Konsens der zu diesem Zeitpunkt Beteiligten zu ermitteln, ob die Festlegungen in meiner Patientenverfügung noch meinem aktuellen Willen entsprechen. Bei unterschiedlichen Meinungen soll in diesen Fällen der Auffassung folgender Personen besondere Bedeutung zukommen: benannte Vertrauenspersonen.

\subsection{Ergänzende Papiere und Aufbewahrung}


Ich habe zusätzlich zur Patientenverfügung eine Vorsorgevollmacht
für Gesundheitsangelegenheiten erteilt und den Inhalt dieser Patientenverfügung mit
den von mir benannten Vertrauenspersonen besprochen. 

Darüberhinaus füge ich bei: ein Blatt mit den derzeitigen Kontaktdaten der hier benannten Personen.

Das Blatt mit den Kontaktdaten enthält auch Angaben über die 
Aufbewahrungsorte.


\section{Regelungen für den Fall des Todes}

\subsection{Beerdigung / Beisetzung}

Ich wünsche Feuerbestattung, soweit dies möglich ist. Nach meinem Tod soll
der Leichnam vollständig verbrannt und die Asche verstreut werden.

\subsection{Organspende}

Ich stimme einer Entnahme meiner Organe nach meinem Tod zu 
Transplantations- oder Forschungszwecken zu.

Ich weise allerdings darauf hin, dass ich seit vielen Jahren an \textit{Morbus Parkinson} erkrankt
bin und dagegen Tabletten nehme. Einige meiner Organe werden daher
vermutlich mit Medikamentenrückständen belastet sein.

Komme ich nach ärztlicher Beurteilung bei einem sich abzeichnenden Hirntod als Organspender in Betracht und müssen dafür ärztliche Maßnahmen durchgeführt werden, die ich in meiner Patientenverfügung ausgeschlossen habe, dann gehen die Bestimmungen in meiner Patientenverfügung vor.


\section{Schluss}

Soweit ich bestimmte Behandlungen wünsche oder ablehne, verzichte ich ausdrücklich auf eine (weitere) ärztliche 
Aufklärung.

Mir ist die Möglichkeit der Änderung und des Widerrufs einer Patientenverfügung bekannt.

Ich bin mir des Inhalts und der Konsequenzen meiner darin getroffenen Entscheidungen bewusst.

Ich habe die Patientenverfügung in eigener Verantwortung und ohne äußeren Druck erstellt.

Ich bin im Vollbesitz meiner geistigen Kräfte.


Ich habe mich vor der Erstellung dieser Patientenverfügung durch meine
Neurologin, \mytrustedthree, beraten lassen.



Diese Patientenverfügung gilt solange, bis ich sie widerrufe.

~

\begin{tabular}{ll}

\textit{Ort und Datum} & \rule[-0.2cm]{10cm}{.3mm} \\

~  & ~ \\

~ & ~ \\

\textit{Unterschrift} &  \rule[-0.2cm]{10cm}{.3mm} \\

\end{tabular}



\section{Ärztliche Aufklärung / Bestätigung der Einwilligungsfähigkeit}

Herr \myfullname, geboren am \mybirthday in \myplaceofbirth, 
wurde von mir bezüglich der möglichen Folgen dieser Patientenverfügung aufgeklärt.

Er war in vollem Umfang einwilligungsfähig.

~

\begin{tabular}{ll}

\textit{Ort und Datum} & \rule[-0.2cm]{10cm}{.3mm} \\

~  & ~ \\

~ & ~ \\

\textit{Unterschrift} & \\

\textit{Stempel des Arztes} & \rule[-0.2cm]{10cm}{.3mm} \\

\end{tabular}

\textit{(Die Einwilligungsfähigkeit kann auch durch
 einen Notar bestätigt werden.)}


\newpage

\section*{Anhang -- Weitere Überlegungen für diese Patientenverfügung}

\subsection*{Quellen, helfende Institutionen}

Eine wesentliche Quelle waren die Erläuterungen, Mustertexte und 
Beispiele für Patientenverfügungen auf der Website des
Bundesministeriums der Justiz und für Verbraucherschutz (Abkürzung \textit{BMJV}). Dies ist der Weblink dorthin: 

\href{https://www.bmjv.de/DE/Themen/VorsorgeUndPatientenrechte/Betreuungsrecht/Betreuungsrecht_node.html}{www.bmjv.de}.

In deren Mustertexten finden sich zahlreiche äußerst interessante Fußnoten 
und Anmerkungen, die beim Textverständnis helfen. Die für mich wichtigsten davon möchte ich hier (im folgenden Abschnitt) darstellen.

Weitere Quellen oder Institutionen waren:

\begin{itemize}

\item \href{http://ethikzentrum.de/}{ethikzentrum.de}

\item \href{https://www.parkinson.ch/fileadmin/public/docs/Patientenverfuegung_deutsch/PV_2013_DE_final.pdf}{www.parkinson.ch -- Parkinson Schweiz -- Muster einer Patientenverfügung}

\item \href{http://www.vorsorgeregister.de/}{www.vorsorgeregister.de --
Zentrales Vorsorgeregister der Bundesnotarkammer}


\end{itemize}

Das Ethikzentrum führt unter anderem Prüfungen von Patientenverfügungen
durch.

Bei der schweizerischen Patientenverfügung sind die rechtlichen
nicht auf Deutschland übertragbar. Bemerkenswert erscheinen 
die auf Parkinson bezogenen Hinweise. Allerdings beziehen sich diese
nicht auf Maßnahmen, sondern fast ausschließlich auf Situationen 
und Organspende.
  
  
Das Vorsorgeregister ist eine gute Anlaufstelle für Archivierungsfragen.   


\subsection*{Einzelanmerkungen}

Auch wenn nicht durch Quotierung gekennzeichnet entstammen
alle Anmerkungen dem Mustertext des BMJV. Bis auf kleine redaktionelle
Anpassungen handelt es sich um wörtliche Übernahmen.

\begin{itemize}

\item Zur vorletzten Anwendungssituation: Dieser Punkt betrifft nur Gehirnschädigungen mit dem Verlust der Fähigkeit, Einsichten zu gewinnen, Entscheidungen
zu treffen und mit anderen Menschen in Kontakt zu treten. Es handelt sich dabei häufig um Zustände von
Dauerbewusstlosigkeit oder um wachkomaähnliche Krankheitsbilder, die mit einem vollständigen oder weitgehenden
Ausfall der Großhirnfunktionen einhergehen. Diese Patienten sind in der Regel unfähig zu bewusstem
Denken, zu gezielten Bewegungen oder zu Kontaktaufnahme mit anderen Menschen, während lebenswichtige
Körperfunktionen wie Atmung, Darm- oder Nierentätigkeit erhalten sind, wie auch möglicherweise die Fähigkeit zu
Empfindungen. Wachkoma-Patienten sind bettlägerig, pflegebedürftig und müssen künstlich mit
Nahrung und Flüssigkeit versorgt werden. In seltenen Fällen können sich auch bei Wachkoma-Patienten nach mehreren
Jahren noch günstige Entwicklungen einstellen, die ein eingeschränkt selbstbestimmtes Leben erlauben. Eine sichere
Voraussage, ob die betroffene Person zu diesen wenigen gehören wird oder zur Mehrzahl derer, die ihr Leben lang als
Pflegefall betreut werden müssen, ist bislang nicht möglich.

\item Zum letzten Punkt der Anwendungssituationen: Dies betrifft Gehirnschädigungen infolge eines weit fortgeschrittenen Hirnabbauprozesses, wie sie am häufigsten bei Demenzerkrankungen (z. B. Alzheimer’sche Erkrankung) eintreten. Im Verlauf der Erkrankung werden die Patienten zunehmend unfähiger, Einsichten zu gewinnen und mit ihrer Umwelt verbal zu kommunizieren, während die Fähigkeit zu Empfindungen erhalten bleibt. Im Spätstadium erkennt der Kranke selbst nahe Angehörige nicht mehr und ist schließlich auch nicht mehr in der Lage, trotz Hilfestellung Nahrung und Flüssigkeit auf natürliche Weise zu sich zu nehmen.

\item Zu den Maßnahmen: Die Äußerung, \enquote{keine lebenserhaltenden Maßnahmen} zu wünschen, stellt jedenfalls für sich genommen nicht die für
eine wirksame Patientenverfügung erforderliche hinreichend konkrete Behandlungsentscheidung dar. Die insoweit
erforderliche Konkretisierung kann aber gegebenenfalls durch die Benennung bestimmter ärztlicher Maßnahmen oder die
Bezugnahme auf ausreichend spezifizierte Krankheiten oder Behandlungssituationen erfolgen. Es spricht folglich
grundsätzlich nichts gegen die Verwendung dieser Formulierung, soweit diese nicht isoliert erfolgt, sondern mit konkreten
Beschreibungen der Behandlungssituationen und spezifizierten medizinischen Maßnahmen kombiniert wird. 

\item Das Stillen von Hunger und Durst als subjektive Empfindungen gehört zu jeder lindernden Therapie. Viele
schwerkranke Menschen haben allerdings kein Hungergefühl; dies gilt praktisch ausnahmslos für Sterbende und
wahrscheinlich auch für Wachkoma-Patientinnen oder -Patienten. Das Durstgefühl ist bei Schwerkranken zwar länger als
das Hungergefühl vorhanden, aber künstliche Flüssigkeitsgabe hat nur sehr begrenzten Einfluss darauf. Viel besser kann
das Durstgefühl durch Anfeuchten der Atemluft und durch fachgerechte Mundpflege gelindert werden. Die Zufuhr großer
Flüssigkeitsmengen bei Sterbenden kann schädlich sein, weil sie unter anderem  zu Atemnotzuständen infolge von
Wasseransammlung in der Lunge führen kann.

\item Palliativmedizin ist die medizinische Fachrichtung, die sich primär um die Beschwerdelinderung und Aufrechterhaltung der Lebensqualität bei Patienten mit unheilbaren Erkrankungen kümmert. Eine palliativmedizinische Indikation setzt daher immer das Ziel der Beschwerdelinderung und nicht das Ziel der Lebensverlängerung voraus.

\item Die Schlussformel zum Verzicht auf weitere Aufklärung \enquote{dient dazu, darauf hinzuweisen, dass der Ersteller der Patientenverfügung unter den beschriebenen Umständen keine weitere ärztliche Aufklärung wünscht. Diese Aussage ist besonders wichtig, da bestimmte ärztliche Eingriffe nur dann wirksam vorgenommen werden dürfen, wenn ein Arzt den Patienten vorher hinreichend über die medizinische Bedeutung und Tragweite der geplanten Maßnahmen, alternative Behandlungsmöglichkeiten und Konsequenzen eines Verzichts aufgeklärt hat. Einer ärztlichen Aufklärung bedarf es nicht, wenn der einwilligungsfähige Patient auf eine ärztliche Aufklärung verzichtet hat. Aus der Patientenverfügung sollte sich ergeben, ob diese Voraussetzungen erfüllt sind.}

\item Eine Patientenverfügung muss schriftlich verfasst  
und eigenhändig durch Namensunterschrift oder durch ein durch Notar beglaubigtes Handzeichen unterzeichnet sein. 

Ein Widerruf kann jederzeit formlos erfolgen.


\end{itemize}

 

\end{document}
