



\newpage

\section*{Anhang: Gesetzestexte}

 \addcontentsline{toc}{section}{Anhang: Gesetzestexte}

\subsection*{Bürgerliches Gesetzbuch (BGB)}


\subsubsection*{§ 1904 -- Genehmigung des Betreuungsgerichts bei ärztlichen Maßnahmen}

\begin{legal}

\item Die Einwilligung des Betreuers in eine Untersuchung des Gesundheitszustands, eine Heilbehandlung oder einen ärztlichen Eingriff bedarf der Genehmigung des Betreuungsgerichts, wenn die begründete Gefahr besteht, dass der Betreute auf Grund der Maßnahme stirbt oder einen schweren und länger dauernden gesundheitlichen Schaden erleidet. Ohne die Genehmigung darf die Maßnahme nur durchgeführt werden, wenn mit dem Aufschub Gefahr verbunden ist.

\item Die Nichteinwilligung oder der Widerruf der Einwilligung des Betreuers in eine Untersuchung des Gesundheitszustands, eine Heilbehandlung oder einen ärztlichen Eingriff bedarf der Genehmigung des Betreuungsgerichts, wenn die Maßnahme medizinisch angezeigt ist und die begründete Gefahr besteht, dass der Betreute auf Grund des Unterbleibens oder des Abbruchs der Maßnahme stirbt oder einen schweren und länger dauernden gesundheitlichen Schaden erleidet.

\item Die Genehmigung nach den Absätzen 1 und 2 ist zu erteilen, wenn die Einwilligung, die Nichteinwilligung oder der Widerruf der Einwilligung dem Willen des Betreuten entspricht.

\item Eine Genehmigung nach den Absätzen 1 und 2 ist nicht erforderlich, wenn zwischen Betreuer und behandelndem Arzt Einvernehmen darüber besteht, dass die Erteilung, die Nichterteilung oder der Widerruf der Einwilligung dem nach § 1901a festgestellten Willen des Betreuten entspricht.

\item Die Absätze 1 bis 4 gelten auch für einen Bevollmächtigten. Er kann in eine der in Absatz 1 Satz 1 oder Absatz 2 genannten Maßnahmen nur einwilligen, nicht einwilligen oder die Einwilligung widerrufen, wenn die Vollmacht diese Maßnahmen ausdrücklich umfasst und schriftlich erteilt ist.

\end{legal}


\textit{Quelle:} \url{https://www.gesetze-im-internet.de/bgb/__1904.html}

\textit{Datum der Übernahme:} 25.09.2018




\subsubsection*{§ 1906 -- Genehmigung des Betreuungsgerichts bei freiheitsentziehender Unterbringung und bei freiheitsentziehenden Maßnahmen}

\begin{legal}

\item  Eine Unterbringung des Betreuten durch den Betreuer, die mit Freiheitsentziehung verbunden ist, ist nur zulässig, solange sie zum Wohl des Betreuten erforderlich ist, weil

\begin{enumerate}
\item auf Grund einer psychischen Krankheit oder geistigen oder seelischen Behinderung des Betreuten die Gefahr besteht, dass er sich selbst tötet oder erheblichen gesundheitlichen Schaden zufügt, oder

\item zur Abwendung eines drohenden erheblichen gesundheitlichen Schadens eine Untersuchung des Gesundheitszustands, eine Heilbehandlung oder ein ärztlicher Eingriff notwendig ist, die Maßnahme ohne die Unterbringung des Betreuten nicht durchgeführt werden kann und der Betreute auf Grund einer psychischen Krankheit oder geistigen oder seelischen Behinderung die Notwendigkeit der Unterbringung nicht erkennen oder nicht nach dieser Einsicht handeln kann.

\end{enumerate}

\item Die Unterbringung ist nur mit Genehmigung des Betreuungsgerichts zulässig. Ohne die Genehmigung ist die Unterbringung nur zulässig, wenn mit dem Aufschub Gefahr verbunden ist; die Genehmigung ist unverzüglich nachzuholen.

\item Der Betreuer hat die Unterbringung zu beenden, wenn ihre Voraussetzungen weggefallen sind. Er hat die Beendigung der Unterbringung dem Betreuungsgericht unverzüglich anzuzeigen.

\item Die Absätze 1 bis 3 gelten entsprechend, wenn dem Betreuten, der sich in einem Krankenhaus, einem Heim oder einer sonstigen Einrichtung aufhält, durch mechanische Vorrichtungen, Medikamente oder auf andere Weise über einen längeren Zeitraum oder regelmäßig die Freiheit entzogen werden soll.

\item Die Unterbringung durch einen Bevollmächtigten und die Einwilligung eines Bevollmächtigten in Maßnahmen nach Absatz 4 setzen voraus, dass die Vollmacht schriftlich erteilt ist und die in den Absätzen 1 und 4 genannten Maßnahmen ausdrücklich umfasst. Im Übrigen gelten die Absätze 1 bis 4 entsprechend.

\end{legal}

\textit{Quelle:} \url{https://www.gesetze-im-internet.de/bgb/__1906.html}

\textit{Datum der Übernahme:} 25.09.2018


\subsubsection*{§ 1906a -- Genehmigung des Betreuungsgerichts bei ärztlichen Zwangsmaßnahmen}

\begin{legal}

\item  Widerspricht eine Untersuchung des Gesundheitszustands, eine Heilbehandlung oder ein ärztlicher Eingriff dem natürlichen Willen des Betreuten (ärztliche Zwangsmaßnahme), so kann der Betreuer in die ärztliche Zwangsmaßnahme nur einwilligen, wenn

\begin{enumerate}
\item die ärztliche Zwangsmaßnahme zum Wohl des Betreuten notwendig ist, um einen drohenden erheblichen gesundheitlichen Schaden abzuwenden,

\item der Betreute auf Grund einer psychischen Krankheit oder einer geistigen oder seelischen Behinderung die Notwendigkeit der ärztlichen Maßnahme nicht erkennen oder nicht nach dieser Einsicht handeln kann,

\item die ärztliche Zwangsmaßnahme dem nach § 1901a zu beachtenden Willen des Betreuten entspricht,

\item zuvor ernsthaft, mit dem nötigen Zeitaufwand und ohne Ausübung unzulässigen Drucks versucht wurde, den Betreuten von der Notwendigkeit der ärztlichen Maßnahme zu überzeugen,

\item der drohende erhebliche gesundheitliche Schaden durch keine andere den Betreuten weniger belastende Maßnahme abgewendet werden kann,

\item der zu erwartende Nutzen der ärztlichen Zwangsmaßnahme die zu erwartenden Beeinträchtigungen deutlich überwiegt und

\item die ärztliche Zwangsmaßnahme im Rahmen eines stationären Aufenthalts in einem Krankenhaus, in dem die gebotene medizinische Versorgung des Betreuten einschließlich einer erforderlichen Nachbehandlung sichergestellt ist, durchgeführt wird.

\end{enumerate}

§ 1846 ist nur anwendbar, wenn der Betreuer an der Erfüllung seiner Pflichten verhindert ist.

\item Die Einwilligung in die ärztliche Zwangsmaßnahme bedarf der Genehmigung des Betreuungsgerichts.

\item Der Betreuer hat die Einwilligung in die ärztliche Zwangsmaßnahme zu widerrufen, wenn ihre Voraussetzungen weggefallen sind. Er hat den Widerruf dem Betreuungsgericht unverzüglich anzuzeigen.

\item Kommt eine ärztliche Zwangsmaßnahme in Betracht, so gilt für die Verbringung des Betreuten gegen seinen natürlichen Willen zu einem stationären Aufenthalt in ein Krankenhaus § 1906 Absatz 1 Nummer 2, Absatz 2 und 3 Satz 1 entsprechend.

\item Die Einwilligung eines Bevollmächtigten in eine ärztliche Zwangsmaßnahme und die Einwilligung in eine Maßnahme nach Absatz 4 setzen voraus, dass die Vollmacht schriftlich erteilt ist und die Einwilligung in diese Maßnahmen ausdrücklich umfasst. Im Übrigen gelten die Absätze 1 bis 3 entsprechend.

\end{legal}

\textit{Quelle:} \url{https://www.gesetze-im-internet.de/bgb/__1906a.html}

\textit{Datum der Übernahme:} 25.09.2018


